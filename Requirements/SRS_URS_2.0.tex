
\documentclass[12pt]{article}
\usepackage{hyperref}
\usepackage{geometry}
\geometry{a4paper, margin=1in}
\usepackage{graphicx}
\usepackage{enumitem}

\title{Software Requirements Specification  and User Requirements Specification  \\ \textbf{SmartScroll: AI-Driven Interactive Learning Platform}}
\author{}
\date{\today}

\begin{document}

\maketitle

\tableofcontents

\newpage

\section{Scope}
The AI-Driven Interactive Learning Platform is designed to enhance modern education by providing personalized learning experiences that support knowledge acquisition, retention, and application. By leveraging AI and interactive tools, the platform addresses key educational challenges such as engagement, personalization, and skill development. The system facilitates the creation of quizzes, flashcards, and summaries, all tailored to individual learning styles and goals. The platform is designed for accessibility, ensuring users can access educational content on both web and mobile devices. The scope includes features for real-time performance tracking, feedback, and gamified learning elements, all aimed at improving students' ability to apply higher-order cognitive skills as outlined by Bloom's revised taxonomy.

\subsection{Challenges Addressed}
The platform addresses key challenges in education, including:
\begin{itemize}
    \item \textbf{Low Retention and Engagement}: Learners often struggle to stay focused and retain information due to monotonous delivery methods.
    \item \textbf{Inadequate Personalization}: One-size-fits-all approaches fail to meet the diverse needs, goals, and learning styles of students.
    \item \textbf{Limited Higher-Order Skill Development}: Traditional methods focus on memorization rather than application, analysis, and evaluation.
    \item \textbf{Lack of Timely Feedback}: Learners need actionable insights to assess progress and improve.
    \item \textbf{Motivation Challenges}: Without gamified elements or collaborative tools, learners may lose interest and motivation.
\end{itemize}

\subsection{Proposed Solution}
The platform provides the following solutions to overcome these challenges:
\begin{itemize}
    \item \textbf{Interactive Content Delivery}: Offers flashcards, quizzes, and gamified exercises to make learning engaging and effective.
    \item \textbf{Personalized AI-Powered Learning Journeys}: Adapts to individual learning preferences, progress, and goals for a tailored experience.
    \item \textbf{Real-Time Feedback and Progress Tracking}: Provides actionable insights to help learners understand their strengths and areas for improvement.
    \item \textbf{Collaborative Learning Tools}: Facilitates group tasks and peer interaction to promote teamwork and engagement.
    \item \textbf{Skill Development Aligned with Bloom's Taxonomy}: Encourages higher-order cognitive skills, such as analysis, evaluation, and creation, through targeted activities.
\end{itemize}


\section{Software Requirements Specification (SRS)}

\subsection{Functional Requirements}
\begin{enumerate}

    \item \textbf{Content Upload}:
        \begin{itemize}
            \item The system shall allow users to upload educational materials in PDF format only.
            \item \textit{Rationale}: Ensures consistent content processing and compatibility with AI models.
            \item \textit{Validation}: Upload valid and invalid PDF files to confirm proper ingestion and error handling.
        \end{itemize}
        
    \item \textbf{AI-Driven Content Generation}:
        \begin{itemize}
            \item The system shall automatically generate quizzes, flashcards, and summaries from uploaded PDF materials.
            \item \textit{Rationale}: Provides immediate and personalized learning aids to enhance user engagement.
            \item \textit{Validation}: Verify the accuracy and relevance of generated outputs using sample input materials.
        \end{itemize}

    \item \textbf{Error Handling for Invalid Inputs}:
        \begin{itemize}
            \item The system shall detect and notify users of invalid or unsupported uploads (e.g., non-PDF formats, corrupted files).
            \item \textit{Rationale}: Prevents system errors and improves user experience by providing clear feedback.
            \item \textit{Validation}: Attempt to upload invalid files and confirm error messages and prevention of processing.
        \end{itemize}

    \item \textbf{Gamification Features}:
        \begin{itemize}
            \item The system shall incorporate gamified elements such as task rewards, leaderboards, and collaborative group challenges.
            \item \textit{Rationale}: Enhances user motivation and engagement through interactive learning.
            \item \textit{Validation}: Simulate gamification workflows and assess user engagement metrics.
        \end{itemize}

    \item \textbf{Progress Tracking and Feedback}:
        \begin{itemize}
            \item The system shall provide real-time performance analysis, including quiz scores, time spent, and learning trends.
            \item The system shall allow users to access and review their past results and feedback.
            \item \textit{Rationale}: Enables self-assessment, targeted improvement, and progress monitoring over time.
            \item \textit{Validation}: Confirm that user dashboards accurately reflect performance metrics and allow historical review.
        \end{itemize}

    \item \textbf{Multi-Language Support}:
        \begin{itemize}
            \item The system shall support German and English interfaces, with the ability to expand to additional languages.
            \item \textit{Rationale}: Increases accessibility for a broader user base.
            \item \textit{Validation}: Test language switching functionality and ensure all interfaces are translated accurately.
        \end{itemize}

    \item \textbf{Accessibility Features}:
    \begin{itemize}
        \item The system shall ensure accessibility by providing a responsive design that functions well across various devices (e.g., desktops, tablets, mobile phones).
        \item \textit{Rationale}: Ensures a seamless user experience for all users, regardless of their device.
        \item \textit{Validation}: Test the platform on multiple devices and screen resolutions to verify responsiveness and usability.
    \end{itemize}

    \item \textbf{User Authentication and Security}:
        \begin{itemize}
            \item The system shall authenticate users using secure methods such as OAuth2.
            \item The system shall restrict unauthorized access to content and data.
            \item \textit{Rationale}: Protects user data and ensures secure access.
            \item \textit{Validation}: Attempt unauthorized access and verify system security mechanisms.
        \end{itemize}

        \item \textbf{Quiz Disclaimer}:
    \begin{itemize}
        \item The system shall display a disclaimer before each generated quiz, informing users that the answers provided by the AI may occasionally be incorrect.
        \item \textit{Rationale}: Ensures transparency and helps manage user expectations regarding the accuracy of AI-generated content.
        \item \textit{Validation}: Verify that the disclaimer is displayed consistently and prominently before accessing quizzes.
    \end{itemize}

\item \textbf{Flagging Mechanism for Incorrect Answers}:
    \begin{itemize}
        \item The system shall allow users to flag quiz answers they believe are incorrect.
        \item \textit{Rationale}: Provides a feedback mechanism to identify and correct errors, improving content quality over time.
        \item \textit{Validation}: Confirm that flagged content is logged and accessible for review by administrators.
    \end{itemize}
        
\end{enumerate}


\subsection{Non-Functional Requirements}
\begin{enumerate}

    \item \textbf{Performance}  
    \textbf{Requirement}: The system shall process uploaded content and generate learning resources within 5 minutes for files up to 100 MB. \\
    \textbf{Rationale}: This ensures timely feedback for users, supporting their workflow and productivity. \\
    \textbf{Validation}: Verify processing times under various load conditions to ensure the system meets the performance target.

    \item \textbf{Reliability}  
    \textbf{Requirement}: The platform shall maintain an uptime of 99.9\% over a 30-day period, excluding scheduled maintenance. \\
    \textbf{Rationale}: Guarantees consistent availability for users, minimizing disruptions to learning activities. \\
    \textbf{Validation}: Monitor platform uptime over a period of 30 days and calculate system reliability.

    \item \textbf{Usability}  
    \textbf{Requirement}: The platform shall provide an intuitive, user-friendly interface that supports diverse user types, ensuring ease of navigation for both students and educators. \\
    \textbf{Rationale}: Promotes user engagement and reduces learning curves, making the platform accessible to a wide audience. \\
    \textbf{Validation}: Conduct usability testing with target users (students and educators) to verify ease of use and satisfaction.

    \item \textbf{Scalability}  
    \textbf{Requirement}: The platform shall support up to 10,000 concurrent users without performance degradation and shall scale to handle additional growth as needed. \\
    \textbf{Rationale}: Ensures the platform can accommodate a growing user base without affecting performance. \\
    \textbf{Validation}: Perform load testing with increasing numbers of concurrent users to ensure scalability.

    \item \textbf{Security}  
    \textbf{Requirement}: The system shall implement end-to-end encryption for all sensitive data (e.g., personal, educational data), using AES-256 for data at rest and SSL/TLS for data in transit. \\
    \textbf{Rationale}: Protects user data and ensures compliance with GDPR and other privacy regulations. \\
    \textbf{Validation}: Conduct security audits to confirm encryption methods are implemented correctly.

    \item \textbf{Maintainability}  
    \textbf{Requirement}: The platform shall be designed for easy maintenance, with modular architecture and clear API documentation to support future updates and integration of new features. \\
    \textbf{Rationale}: Facilitates long-term system upkeep and adaptability to changing needs. \\
    \textbf{Validation}: Review system documentation and architecture for modularity and ease of maintenance.

    \item \textbf{Compliance}  
    \textbf{Requirement}: The platform shall comply with relevant data privacy regulations, such as GDPR, and applicable accessibility standards (e.g., WCAG 2.1). \\
    \textbf{Rationale}: Ensures the system meets legal requirements and is accessible to all users. \\
    \textbf{Validation}: Perform legal compliance checks and accessibility testing to verify adherence to GDPR and WCAG standards.

    \item \textbf{Fault Tolerance}  
    \textbf{Requirement}: The system shall recover gracefully from failures, ensuring no data loss and minimal service interruption in the event of hardware or software failure. \\
    \textbf{Rationale}: Enhances system resilience and ensures uninterrupted service for users. \\
    \textbf{Validation}: Test recovery mechanisms by simulating hardware or software failures and verifying data integrity and uptime restoration.

    \item \textbf{Transparency in AI-Generated Content}  
    \textbf{Requirement}: The platform shall notify users of potential inaccuracies in AI-generated quizzes to ensure informed use of the content. \\
    \textbf{Rationale}: Builds trust by being transparent about the limitations of AI technology and ensuring users are aware of potential errors. \\
    \textbf{Validation}: Verify that disclaimers are clear, concise, and visible during user interactions with quizzes. Ensure users are prompted to acknowledge this information before starting the quiz.

\end{enumerate}

\subsection{System Architecture and Design}
The AI-Driven Interactive Learning Platform is built on a **3-tier architecture**, a proven design pattern that promotes scalability, maintainability, and separation of concerns. The system is divided into three distinct layers, each with a specific responsibility:

\begin{itemize}
    \item \textbf{Presentation Layer}: This layer handles the user interface (UI), providing interaction points for  students. The UI is built using web technologies ensuring the platform is accessible across devices (desktop, tablet, mobile).
    \item \textbf{Business Logic Layer}: This layer is the core of the platform, implementing the business logic and functionality of the system. Key features include AI-driven content generation, user progress tracking, and personalized learning paths.
    \item \textbf{Data Layer}: This layer manages data storage, retrieval, and manipulation. It stores user data, educational materials, and system logs in databases. The system ensures data consistency and reliability through proper data management and backup protocols.
\end{itemize}

This 3-tier architecture allows the platform to be flexible and scalable, enabling independent scaling of each layer based on system demands. The separation of concerns also makes maintenance easier and allows for focused development efforts in each layer, ensuring a more manageable codebase. For more detailed information on the architecture, please refer to the **Software Architecture and Design Document**.



\newpage

\section{User Requirements Specification (URS)}

\subsection{Business Need}
The AI-Driven Interactive Learning Platform is developed to address critical challenges in modern education, such as low knowledge retention, limited engagement, and the lack of personalized learning experiences. Traditional methods often fail to engage students effectively, resulting in low retention and limited application of learned skills. The platform offers personalized, interactive tools that cater to the diverse learning needs of students, enabling them to engage deeply with educational content. By incorporating real-time feedback, gamification, and AI-driven insights, the platform fosters student motivation, promotes deeper learning, and enhances long-term retention. The platform also integrates mechanisms to notify users about potential inaccuracies in AI-generated content, thereby ensuring trust and transparency in its output.


\subsection{Goals}
The platform aims to achieve the following goals:
\begin{enumerate}
    \item \textbf{Efficient Knowledge Retention}: Improve retention rates by providing interactive and personalized learning tools like quizzes, flashcards, and summaries.
    \item \textbf{Higher-Order Learning}: Encourage students to apply, analyze, and evaluate materials to develop critical thinking and problem-solving skills.
    \item \textbf{Engagement}: Enhance student motivation through gamification, rewards, and collaborative tools such as study groups and peer interactions.
    \item \textbf{Transparency}: Notify users about potential inaccuracies in AI-generated answers, allowing them to verify information independently.
    \item \textbf{Accessibility}: Ensure the platform is available across various devices (web and mobile), making learning resources accessible anywhere, anytime.
\end{enumerate}

\subsection{Stakeholders}
The following stakeholders are involved in the development and use of the AI-Driven Interactive Learning Platform:

\begin{itemize}
    \item \textbf{Primary Users (Students)}:  
    Undergraduate and postgraduate students who use the platform to upload materials, generate quizzes, track progress, and engage with gamified learning features to improve their study efficiency.

    \item \textbf{Secondary Users (Professors and Teaching Assistants)}:  
    Professors and Teaching Assistants who monitor student progress, provide feedback, and guide students through their learning journeys using the platform’s tools.

    \item \textbf{System Administrators}:  
    Individuals responsible for maintaining the platform’s infrastructure, ensuring availability, security, and resolving any technical issues.

    \item \textbf{Development Team}:  
    Engineers and developers responsible for building, testing, and maintaining the platform, ensuring it meets user and business needs.
\end{itemize}




\subsection{User Expectations}
The platform is designed with the following user expectations in mind:
\begin{itemize}
    \item \textbf{Accurate and Reliable Content}: Users expect quizzes, flashcards, and summaries to be accurate and relevant to the uploaded material. They should also be notified about any potential inaccuracies in AI-generated content.
    \item \textbf{Ease of Use}: The platform should have an intuitive and user-friendly interface, enabling students to easily upload materials, access resources, and track progress.
    \item \textbf{Timely Feedback}: Students expect the system to process uploaded materials and generate resources quickly (e.g., within 5 minutes for files up to 100 MB).
    \item \textbf{Personalization}: Users expect learning resources to be tailored to their preferences, goals, and progress, offering a personalized learning journey.
    \item \textbf{Secure Data Handling}: Users expect their data to be stored and managed securely, ensuring compliance with privacy standards such as GDPR.
\end{itemize}

\subsection{User Stories}
\begin{itemize}
    \item \textbf{As a student}, I want to upload educational materials so that I can generate personalized quizzes and summaries.
        \begin{itemize}
            \item \textit{Acceptance Criteria}: Uploaded content is processed within 5 minutes, and generated resources are accessible from the dashboard.
        \end{itemize}
    
    \item \textbf{As a student}, I want to track my learning progress so that I can focus on areas where I need improvement.
        \begin{itemize}
            \item \textit{Acceptance Criteria}: The dashboard shows study time, quiz scores, and performance trends with the ability to filter by date and topic.
        \end{itemize}
    
    \item \textbf{As a student}, I want to set learning preferences so that the system can tailor quizzes, flashcards, and summaries to my personal learning style and goals.
        \begin{itemize}
            \item \textit{Acceptance Criteria}: Users can choose learning modes (e.g., quiz frequency, summary length), and preferences are saved for future use.
        \end{itemize}

    \item \textbf{As a student}, I want to review past quiz results so that I can track my improvements over time.
        \begin{itemize}
            \item \textit{Acceptance Criteria}: Users can view past quiz results, including correct answers and performance analytics, with drill-down options for each question.
        \end{itemize}
    
    \item \textbf{As a student}, I want to be notified if AI-generated answers in quizzes could potentially be inaccurate so that I can verify the information independently.
        \begin{itemize}
            \item \textit{Acceptance Criteria}: A disclaimer is displayed before accessing quizzes, stating the potential for inaccuracies in AI-generated answers.
        \end{itemize}
\end{itemize}

\section{Conclusion}
This document now integrates a refined scope, expanded requirements, and alignment with cognitive frameworks to ensure the SmartScroll platform is engaging, impactful, and aligned with modern educational needs. By incorporating user feedback mechanisms and transparency about AI limitations, the platform ensures trust and reliability for its users.


\end{document}
